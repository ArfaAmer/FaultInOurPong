\documentclass{article}

\usepackage{tabularx}
\usepackage{booktabs}
\usepackage{comment}

\title{SE 3XA3: Problem Statement\\FaultInOurPong}

\author{Lab 01, Team 03, Team Name!!!!!!!!!!!!!!!!!!!!
		\\ Adwity Sharma - sharma78
		\\ Arfa Butt - buttaa3
		\\ Jie Luo - luoj3
}

\date{}

\begin{document}

\begin{table}[hp]
\caption{Revision History} \label{TblRevisionHistory}
\begin{tabularx}{\textwidth}{llX}
\toprule
\textbf{Date} & \textbf{Developer(s)} & \textbf{Change}\\
\midrule
September 23th, 2016 & Adwity, Arfa, Jie & Created an initial problem statement for the project Revision 0\\
\bottomrule
\end{tabularx}
\end{table}

\newpage

\maketitle

\noindent\textbf{Introduction}\\

The development of technology has brought playground indoors and dramatically increased the need for games fit for all ages. It is no wonder that the popular ping pong, also known as table tennis has its electronic version. Our team has decided to re-design the iconic ping pong game by adding a multi-player mode that can be enjoyed by people of all ages to pass time more easily. \\

\noindent\textbf{Importance}\\

Earlier versions of the ping pong game were available to users with purchase of a computer running Microsoft windows but later versions do not have it pre-installed. Our game, which will be developed into a third-party version, provides modes for both individual and multi-player. The multi-player version of the game can be enjoyed among friends, while the single player option individual pastime. Our design will provide users with an opportunity to enjoy this classic game with a little twist added to it. \\

\noindent\textbf{Context of the Problem}\\

The developing team, game players, the instructor, teaching assistants and the future developers are the stakeholders of this project. We are the developers and we are responsible for the delivery and maintenance of the project throughout its lifespan. Although this project does not have very strict guidelines in terms of the model of the project, it still does follow some set criteria set by the professor and the SE3XA3 team, thus they are the clients of this project. This game does not have any age limitations, thus anyone who wants to play this game will fall under the user category. This game will require a Java running platform to run. It can be enjoyed in any environment, given the user has time to play this game and a computer to play it on. Future developers(including us) are also stakeholders for maintenance and further development.



\end{document}