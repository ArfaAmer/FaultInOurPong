\documentclass{article}
\usepackage{tabularx}
\usepackage{booktabs}
\usepackage{comment}
\usepackage{indentfirst}
\usepackage{color}
\title{SE 3XA3: Problem Statement\\FaultInOurPong}
\author{Lab 01, Team 03, Pongthusiastics
		\\ Adwity Sharma - sharma78
		\\ Arfa Butt - buttaa3
		\\ Jie Luo - luoj3
}
\date{}
\begin{document}
\begin{table}[hp]
\caption{Revision History} \label{TblRevisionHistory}
\begin{tabularx}{\textwidth}{llX}
\toprule
\textbf{Date} & \textbf{Developer(s)} & \textbf{Change}\\
\midrule
September 23th & Adwity, Arfa, Jie & Created an initial problem statement for
the project Revision 0\\
September 26th & Jie & Improved on the latex syntax\\
December 6th & Arfa & Improved on sentence structure and updated the document for Revision 1\\
\bottomrule
\end{tabularx}
\end{table}
\newpage
\maketitle
\section{Introduction}
The development of technology has brought playground indoors and \textcolor{red}{has} dramatically
increased the need for games \textcolor{red}{that are} fit for all ages. It is no wonder that the popular
ping pong, also known as table tennis has its electronic version. Our team has
decided to re-design the iconic ping pong game by adding a \textcolor{red}{new bomb} mode
that can be enjoyed by people of all ages to pass time more easily. 
\section{Importance}
Earlier versions of the ping pong game were available to users with purchase of
a computer running Microsoft windows but later versions do not have it
pre-installed. Our game, which will be developed into a third-party version,
provides \textcolor{red}{two seperate} modes for \textcolor{red}{the user. The new bomb version enhances
the difficulty level of the game, and is designed for the users who want to experience a more challenging version of this iconic game, 
where as the classic single player mode remains the same, with a few in-game options added to it}. 
Our design will provide users with an opportunity to enjoy this classic
game with a little twist added to it. 
\section{Context of the Problem}
The developing team, game players, the instructor, teaching assistants and the
future developers are the stakeholders of this project. We are the developers
and are responsible for the delivery and maintenance of the project
throughout its lifespan. Although this project does not have very strict
guidelines in terms of the model of the project, it still does follow some set
criteria set by the professor and the SE3XA3 team, thus they are the clients of
this project. This game does not have any age limitations, thus anyone who wants
to play this game will fall under the user category. This game will require a
Java running platform to run. It can be enjoyed in any environment, given the
user has time to play this game and a computer to play it on. Future
developers(including us) are also stakeholders for maintenance and further
development.\\
\end{document}