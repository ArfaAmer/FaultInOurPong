\documentclass{article}
 \usepackage{booktabs} 
\usepackage{tabularx}
\usepackage{indentfirst}
\usepackage{color}
 \title{SE 3XA3: Development Plan\\Title of Project} 
\author{Lab 01, Team 03, Pongthusiastics 		
\\ Adwity Sharma - sharma78 		
\\ Arfa Butt - buttaa3 	
	\\ Jie Luo - luoj3 }
 \date{} 
\begin{document} 
\begin{table}[hp]
 \caption{Revision History} \label{TblRevisionHistory} 
\begin{tabularx}{\textwidth}{llX} 
\toprule
 \textbf{Date} & \textbf{Developer(s)} & \textbf{Change}\\ 
\midrule 
September 24th & Jie  & Created a development plan for Revision 0\\ 
September 29th & Jie,Adwity,Arfa & Discussed development plan for Revision 0\\ 
September 30th & Arfa,Adwity & Modified development plan for Revision 0\\ 
September 30th & Jie  & Modified development plan for Revision 0\\ 
December 7th & Arfa & Updated document for Revision 1\\
\bottomrule
 \end{tabularx} 
\end{table} 
\newpage 
\maketitle


\section {Team Meeting Plan} 

Team members will be meeting on a daily basis. Two types of meeting will be utilized: the face-to-face meeting and the on-line meeting. The meetings are scheduled based on team members' course schedules and their availabilities. The on-line meeting plan is listed in the following section, and the face-to-face meeting plan is described below.\\

\begin{tabular}{|p{3cm}|p{8cm}|}
\hline
\textbf{Details}    & \textbf{Descriptions}                                                                                                                                                                                                                                                                                                                                                                                                                                \\\hline
When             & {There should be a face-to-face meeting scheduled on every Wednesday from 2:30 p.m. to 5:30 p.m, roughly three hours in total. The number of hours for meeting could vary, depending on the meeting progress.}                                                                                                                                                                                                                                                      \\\hline
Where            & The meeting location would be the Health Science Library (or any other library at McMaster University)                                                                                                                                                                                                                                                                                                                                                                                \\\hline
Frequency        & Once a week or more depending on the need                                                                                                                                                                                                                                                                                                                                                                                                           \\\hline
Roles            & \begin{tabular}[c]{@{}l@{}}Team leader - Jean Luo\\ Schedule manager - Arfa Butt\\ Note taker - Adwity Sharma\end{tabular}                                                                                                                                                                                                                                                                                                                          \\\hline
Rules for agenda & If anyone cannot attend a pre-scheduled meeting, he/she should notify the team at least 3 hours before so the meeting can be rescheduled and must bring food to the next meeting. Each team member should be prepared for the meeting, so that during the meeting, each member would show what they have done since the last meeting, any problems encountered during the project work, and any plans that should be finished before next meeting.  \\ \hline
\end{tabular}




\section {Team Communication Plan} 

In addition to meeting face-to-face, team members also agreed to share their personal contact for on-line meeting. Several on-line applications will be utilized for contacts, discussions, and file-sharing.\\ 

\begin{tabular}{|p{3cm}|p{8cm}|}
\hline
\textbf{Tools}    & \textbf{Descriptions}                                                                                                                                                                                                                                                                                                                                                                                                                                \\\hline
Git        & It will be used to pull or push files from the repository; it will also be used to get notifications about any current issue with the on going project by the instructor or the course coordinators. \\\hline
Facebook    & The xa3 group chat \textcolor{red}{will be used} to discuss any implementation problems and administrative details regarding the project. \\\hline
Phones/texts   & They will be used to contact team members; they will be mainly used as emergency contact. \\\hline
Google Docs   & It will be used to work together on documentations, before transferring them into the Latex format. \\\hline
\end{tabular}



\section  {Team Member Roles}

Each team member is assigned at least one role for this project. The team leader will be in charge of the project as a whole, including guiding the members in the progress of development, delegating tasks to members properly, and maintaining healthy group dynamics. An expert on technology will be focusing on the software development of the project, and he/she will be helping other team members on programming problems/issues. A documentation expert will be managing the tasks on documentations, and ensuring the good quality of the documentations at submission.\\

\begin{tabular}{|p{3cm}|p{8cm}|}
\hline
\textbf{Members}    & \textbf{Roles}                                                                                                                                                                                                                                                                                                                                                                                                                                \\\hline
Arfa Amer Butt       & Expert on technology \\\hline
Jean Luo   & Team leader, Git and Latex expert \\\hline
Adwity Sharma   & Documentation expert \\\hline
\end{tabular}

\section  { Git Workflow Plan}

The centralized flow plan will be used on Git. A master branch is created on GitLab for file sharing and submissions. When working on the project, each team members will be pulling or committing files from the master branch. The commits will be identified by Mac IDs and commit messages. Labels are considered as tags for the submissions. At the end of the semester, this project will be submitted as a milestone.\\

\section  {Proof of Concept Demonstration Plan}

Testing will be difficult because it is a game and self testing could turn out to be a challenge - we can change the parameters (values and data) in the model part of the program. For example to see if the counting of score is taking place properly we can program it in the model in such a way that if a user misses the pong in the first turn, we can use assert statements to check if the miss has been recorded as it was implemented to be.\\

For the implementation of the game we are planning to add a \textcolor{red}{bomb-mode option, so that could turn out to be a challenge - bomb-mode will require the implementation of two balls and keeping track of them at the same time might prove challenging. 
Since we are also planning on adding an in-game options panel, implementing the use of a mouse listener (for the in-game options) and a key listener (for the actual game) together could prove difficult.}\\

\textcolor{red}{No external library is being used so that will not be an issue.}\\

\textcolor{red}{For the proof of concept demonstration, we will provide a prototype of the ping pong game, with two working modes; Single player and Advanced player with the bomb. The movements of the paddles and the in-game options will be implemented 
and will be controlled by the user through the following commands:
\begin{itemize}
\item User's paddle will move left and right through the 'left' and 'right' keys on the keyboard, respectively.
\item User will choose to pause, resume, save and exit the game by pressing on the corresponding button with their mouse.
\end{itemize}
Basic features and commands of the game will be demonstrated, while detailed testing will be done at a later date.}

\section {Technology}

\begin{tabular}{|p{5cm}|p{8cm}|}
\hline
\textbf{Tools}    & \textbf{Details}                                                                                                                                                                                                                                                                                                                                                                                                                                \\\hline
Programming language       & Java\\\hline
IDE      & Eclipse Java IDE\\\hline
Testing framework       & Junit testing, integration testing, black box testing, and testing through user surveys \\\hline
Document generation   & Latex, Google docs for working concurrently, Microsoft word to work on rough \\\hline
\end{tabular}\bigskip

For the game to work one will require a Java environment (JRE), which could be a challenge for the intended users. Team members will try to create an android friendly version of the game\textcolor{red}{. However because of time constrains, this part of 
the implementation will be scheduled after the end of the course.} \\

\section {Coding Style}
The Google Java Style will be the guide for this project. To be precise, the codes will adhere the Google Java Style in terms of its the format, class structures, naming conventions, and any other features pointed out in the guide.\\

The more detailed information on the Google Java Style Guide can be found here: https://google.github.io/styleguide/javaguide.html . \\


\section {Project Schedule}

The project schedule can be found here: \\
https://gitlab.cas.mcmaster.ca/Group3/FaultInOurPong/blob/master/ProjectSchedule/GanttChart.pdf . 


\section {Project Review} 

\end{document}